%%%%%%%%%%%%%%%%%%%%%%%%%%%%%%%%%%%%%%%%%%%%%%%%%%%%%%%%%%%
% Obere Titelmakros. Editieren Sie diese Datei nur, wenn
% Sie sich ABSOLUT sicher sind, was Sie da tun!!!
% (Z.B. zum Abaendern der BA-Vorlage in eine MA-Vorlage)
% Uni Duesseldorf
% Lehrstuhl fuer Datenbanken und Informationssysteme
% Version 2.2 - 2.3.2010
%%%%%%%%%%%%%%%%%%%%%%%%%%%%%%%%%%%%%%%%%%%%%%%%%%%%%%%%%%%
\newcommand{\AN}{twoside}
\newcommand{\AUS}{}
%\newcommand{\englisch}{}
%\newcommand{\deutsch}{\usepackage[german]{babel}}

%% Die folgenden auskommentierten Optionen dienen der automatischen
%% Erkennung des Latex-Kompilers und dem Setzen der davon abh�ngigen
%% Einstellungen. Bei Problem z.B. mit dem Einbinden von verschiedenen
%% Grafiktypen bei Verwendung von PdfLatex oder Latex, einfach die
%% verschiedenen \usepackage(s) ausprobieren. (Mit diesen Einstellungen
%% funktionierte diese Vorlage bei der Verwenundg von latex.exe als
%% Kompiler bei den meisten Studierenden.)

%\newif\ifpdf \ifx\pdfoutput\undefined
%\pdffalse % we are not running pdflatex
%\else
%\pdfoutput=1 % we are running pdflatex
%\pdfcompresslevel=9 % compression level for text and image;
%\pdftrue \fi

\documentclass[11pt,a4paper, \zweiseitig]{article}



%\usepackage[iso]{umlaute}
%\usepackage[latin1]{inputenc}
\usepackage{palatino} % palatino Schriftart
%\usepackage{makeidx} % um ein Index zu erstellen
\usepackage{tocbibind}
\usepackage[T1]{fontenc} %fuer richtige Trennung bei Umlauten
\usepackage{fancybox} % fuer die Rahmen
\usepackage{shortvrb}
\usepackage{ifthen}
%\ifthenelse{\equal{\sprache}{deutsch}}{\usepackage[ngerman]{babel}}{}
\usepackage[utf8]{inputenc}
\usepackage{lmodern} 
\usepackage{amsmath}
\usepackage{oldgerm}
\usepackage{amssymb}
\usepackage{pdfpages}
\usepackage{hyperref}
%\usepackage{fancyheadings}
\usepackage{fancyhdr}
\usepackage{amsfonts}
\usepackage{amsthm}
\usepackage{color}
%\usepackage{stmaryrd}
\usepackage{graphicx}
\usepackage{nomencl}
\usepackage[normalem]{ulem} 
\usepackage{verbatim}
\usepackage{ulsy}
\usepackage{tikz}
\usepackage{pgf}
\usepackage{bbding}
\usepackage{nicefrac}
\usepackage{stmaryrd}
\usepackage{multirow}
% \usepackage[disable]{todonotes}				% alle todo-Anmerkungen ausblenden
\usepackage[german, color=yellow!40, colorinlistoftodos, textsize=footnotesize, shadow]{todonotes}
\usetikzlibrary{arrows,automata,petri,shapes,snakes}


\newcommand{\markup}[1]{\uline{#1}}
% Befehl umbenennen in abk
\let\abk\nomenclature

\usepackage{a4wide} % ganze A4 Weite verwenden

%\ifpdf
%\usepackage[pdftex,xdvi]{graphicx}
%\usepackage{thumbpdf} %thumbs fuer Pdf
%\usepackage[pdfstartview=FitV]{hyperref} %anklickbares Inhaltsverzeichnis
%\else
%\usepackage[dvips,xdvi]{graphicx}
%\usepackage{hyperref} %anklickbares Inhaltsverzeichnis
%\fi

%%%%%%%%%%%%%%%%%%%%%%% Massangaben fuer die Arbeit %%%%%%%%%%%%%%%
\setlength{\textwidth}{15cm}

\setlength{\oddsidemargin}{35mm}
\setlength{\evensidemargin}{25mm}

\addtolength{\oddsidemargin}{-1in}
\addtolength{\evensidemargin}{-1in}

%\makeindex
% Umgebungen f"ur S�tze usw.
\newtheorem{bemerkung*}{Bemerkung}
\newtheorem{defi}{Definition}
\newtheorem{defi*}{Definition}
\newtheorem{bezeichnungen}{Bezeichnungen}
\newtheorem{fakt}{Fakt}
\newtheorem{beispiel}{Beispiel}
\newtheorem{bsp}{Beispiel}
\newtheorem{beispiel*}{Beispiel}
\newtheorem{satz}{Satz}
\newtheorem{satz*}{Theorem}
\newtheorem{lem}{Lemma}
\newtheorem{idee}{Idee}
\newtheorem{corollary}{Korollar}

%definition of new commands
\newcommand{\DGEF}{\text{\textbf{DGEF}}}
\newcommand{\vsp}{\vspace{3mm}}
\newcommand{\impl}[1]{\overset{\text{#1}}{\implies}}
\newcommand{\notimplleft}[1]{\overset{\text{#1}}{\not\Leftarrow}}
\renewcommand{\figurename}{Abbildung}
\begin{document}

%\setcounter{secnumdepth}{4} %Nummerieren bis in die 4. Ebene
%\setcounter{tocdepth}{4} %Inhaltsverzeichnis bis zur 4. Ebene

\pagestyle{headings}
\thispagestyle{empty}
\sloppy % LaTeX ist dann nicht so streng mit der Silbentrennung
%\MakeShortVerb{\�}

\parindent0mm
\parskip0.5em


{
\textwidth170mm 
\oddsidemargin30mm 
\evensidemargin30mm 
\addtolength{\oddsidemargin}{-1in}
\addtolength{\evensidemargin}{-1in}

\parskip0pt plus2pt

% Die Raender muessen eventuell fuer jeden Drucker individuell eingestellt
% werden. Dazu sind die Werte fuer die Abstaende `\oben' und `\links' zu
% aendern, die von mir auf jeweils 0mm eingestellt wurden.

%\newlength{\links} \setlength{\links}{10mm}  % hier abzuaendern
%\addtolength{\oddsidemargin}{\links}
%\addtolength{\evensidemargin}{\links}

\begin{titlepage}
\vspace*{-1.5cm}
  \raisebox{17mm}{
    \begin{minipage}[t]{70mm}
      \begin{center}
        %\selectlanguage{german}
        {\Large INSTITUT FÜR INFORMATIK\\}
        {\normalsize
          Algorithmen und Datenstrukturen
\\
        }
        \vspace{3mm}
        {\small Universitätsstr. 1 \hspace{5ex} D--40225 Düsseldorf\\}
     \end{center}
    \end{minipage}
  }
  \hfill
  \includegraphics[width=130pt]{HHU_Logo}
  \vspace{14em}

% Titel
  \begin{center}
      	\baselineskip=55pt
    	\textbf{\huge \titel}
  	 	\baselineskip=0 pt
   \end{center}

  %\vspace{7em}

\vfill

% Autor
  \begin{center}
    \textbf{\Large
      \bearbeiter
    }
  \end{center}

  \vspace{35mm}
 
% Pr�fungsordnungs-Angaben
  \begin{center}
    %\selectlanguage{german}
    
%%%%%%%%%%%%%%%%%%%%%%%%%%%%%%%%%%%%%%%%%%%%%%%%%%%%%%%%%%%%%%%%%%%%%%%%%
% Ja, richtig, hier kann die BA-Vorlage zur MA-Vorlage gemacht werden...
%%%%%%%%%%%%%%%%%%%%%%%%%%%%%%%%%%%%%%%%%%%%%%%%%%%%%%%%%%%%%%%%%%%%%%%%%
    {\Large Masterarbeit}

    \vspace{2em}

    \begin{tabular}[t]{ll}
      Beginn der Arbeit:& \beginndatum \\
      Abgabe der Arbeit:& \abgabedatum \\
      Gutachter:         & \erstgutachter \\
                         & \zweitgutachter \\
    \end{tabular}
  \end{center}

\end{titlepage}

}

%%%%%%%%%%%%%%%%%%%%%%%%%%%%%%%%%%%%%%%%%%%%%%%%%%%%%%%%%%%%%%%%%%%%%
\clearpage
\begin{titlepage}
  ~                % eine leere Seite hinter dem Deckblatt
\end{titlepage}
%%%%%%%%%%%%%%%%%%%%%%%%%%%%%%%%%%%%%%%%%%%%%%%%%%%%%%%%%%%%%%%%%%%%%
\clearpage
\begin{titlepage}
\vspace*{\fill}

\section*{Erklärung}

%%%%%%%%%%%%%%%%%%%%%%%%%%%%%%%%%%%%%%%%%%%%%%%%%%%%%%%%%%%
% Und hier ebenfalls ggf. BA durch MA ersetzen...
%%%%%%%%%%%%%%%%%%%%%%%%%%%%%%%%%%%%%%%%%%%%%%%%%%%%%%%%%%%

Hiermit versichere ich, dass ich diese Masterarbeit
selbstständig verfasst habe. Ich habe dazu keine anderen als die
angegebenen Quellen und Hilfsmittel verwendet.

\vspace{25 mm}

\begin{tabular}{lc}
Düsseldorf, den \abgabedatum \hspace*{2cm} & \underline{\hspace{6cm}}\\
& \bearbeiter
\end{tabular}

\vspace*{\fill}
\end{titlepage}

%%%%%%%%%%%%%%%%%%%%%%%%%%%%%%%%%%%%%%%%%%%%%%%%%%%%%%%%%%%%%%%%%%%%%
% Leerseite bei zweiseitigem Druck
%%%%%%%%%%%%%%%%%%%%%%%%%%%%%%%%%%%%%%%%%%%%%%%%%%%%%%%%%%%%%%%%%%%%%

\ifthenelse{\equal{\zweiseitig}{twoside}}{\clearpage\begin{titlepage}
~\end{titlepage}}{}

%%%%%%%%%%%%%%%%%%%%%%%%%%%%%%%%%%%%%%%%%%%%%%%%%%%%%%%%%%%%%%%%%%%%%
\clearpage
\begin{titlepage}

\section*{\ifthenelse{\equal{\sprache}{deutsch}}{Zusammenfassung}{Abstract}}
